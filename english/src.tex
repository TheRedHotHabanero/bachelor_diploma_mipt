\section{Motivation & Problem Statement}

In modern programming, high-level programming languages are particularly
significant, as they allow developers to focus on application
logic rather than memory management details or other low-level
tasks. However, such languages often encounter performance issues,
especially in the context of large and complex projects. This compels
people to create new high-level programming languages.

In this work, we are exploring a new programming language that is actively
developing as a fast alternative to the widely used TypeScript.
The main goal of developing this language is to create an extended
and more performant version of TypeScript. It is important to maintain
compatibility with TypeScript to facilitate the transition of existing
projects and the training of new developers.

\textbf{ }
\section{Aim of the work}
The goal of this Bachelor thesis is to develop a code analysis and warning system for the
selected high-level programming language to enhance its performance and speed
up the execution of applications written in it. Additionally, to
implement a system analogous to Clang Tidy for selectively disabling
selected checks directly in the source code.


\textbf{ }
\textbf{Research Tasks:}

\begin{enumerate}
    \item Study of existing solutions.
    \item Development of a code analysis system at the compilation stage.
    \item Analysis of the current specification of the target programming language for potentially slow language constructs and functionality, data collection for further testing.
    \item Provide a correction option that speeds up the application, correct from the perspective of the selected programming language, for each language unit among the proposed ones.
    \item Test each proposal: measure the application's performance before and after the proposed corrections.
    \item Implement the corresponding verification in the analysis system after confirming positive test results.
    \item Develop a system for selectively and collectively disabling checks selected by the developer, supporting the most popular usage scenarios based on data obtained from studying existing solutions.
    \item Support the ability to analyze in a multi-file build system.
\end{enumerate}

The research objectives are considered achieved when at least five proposals are identified and implemented,
speeding up the application by an average of at least 5\%. Additionally, the development
of a check disabling system, successful support of project build, and passing testing
consisting of some potential scenarios of applying the proposed solutions are required.
