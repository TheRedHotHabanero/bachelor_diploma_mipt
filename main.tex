\documentclass{mipt-thesis-bs}

\title{Исследование и разработка системы анализа кода для повышения производительности программ на языке программирования высокого уровня}
\author{Лирисман К.\,С.}
\supervisor{Гаврин Е.\,А.}
\groupnum{Б01-008}
\faculty{Физтех-школа радиотехники и компьютерных технологий}
\department{Кафедра микропроцессорных технологий в телекоммуникационных сетях и вычислительных системах}

\frontmatter
\begin{document}
\titlecontents
\chapter{Аннотация}

В настоящий момент активно развивается статически типизированный управляемый язык
программирования, являющийся расширенной и более быстрой версией языка $TypeScript$ (далее $TS$).
Основная идея разработки спецификации и компилятора этого языка программирования — сделать его максимально
похожим на $TS$ для упрощения перехода будущих разработчиков между $TS$ и
выбранным для исследования языком, а также для ускорения переписывания существующих на $TS$ приложений.
Таким образом, между целевым языком и $TS$ формируется общая часть, - корректная с точки зрения $TS$.
Оставшуюся часть называют не поддерживаемой $TS$.

В данной работе предлагается выделить из целевого языка программирования (далее ЯП) подмножество,
наиболее выгодное с точки зрения производительности,
в том числе за счёт отсечения удобного для разработчиков, но медленного функционала. Таким образом, пока сам язык неминуемо
развивается в сторону общности с $TS$`ом ради легкости перехода, предложенная система помогает держать фокус
на производительности.

Анализ происходит на этапе компиляции исходного кода и предлагает включение желаемых
проверок группами или по отдельности путем добавления флагов компиляции. Реализовано
7 типов проверки, опирающихся на спецификацию выбранного языка
программирования высокого уровня и реализацию его компилятора. К общим с $TS$ проверкам относятся:
неявная упаковка и распаковка, ускорение проверок равенства, запрет инструкций верхнего уровня, исключая
классы и функции и другие. К несовместимым с $TS$ относятся: использование корутин вместо асинхронных функций,
предложение использовать модификатор, ограничивающий наследование классов и методов в случае, если на
момент проверок у них нет потомков.
Предлагается использование двух режимов работы проверяющей системы — в состоянии
предупреждений, а именно, предложений, не обязывающих разработчика к исправлением замечаний и не
влияющих на результат работы программы, и в состоянии, приводящему к ошибке компиляции при ненулевом количестве
предложений для последующих исправлений пользователем.
Предложенная система формулирует рекомендации для повышения производительности. Однако
существуют сценарии использования, когда разработчикам необходимо отключение этих проверок.
С помощью произведенной системы регулирования режимов проверок пользователи целевого языка
могут активировать или деактивировать любую опцию по одной или включать группами.
Также можно снять с проверки определенные строки или целые частей кода - всё это реализуемо с помощью
предложенного аналога системы $Clang-Tidy$, позволяющей точечное отключение проверок непосредственно
в исходном коде программы в виде многострочных или однострочных комментариев.
Таким образом, данная система значительно повышает скорость работы и время запуска
приложения на выбранном статически типизированном языке высокого уровня, позволяет разработчикам
простыми методами улучшить их код и избежать некоторых ошибок.

\mainmatter


\chapter{Введение}

В современном мире существует огромное множество языков программирования.
Часть из этих языков программирования одной из своих целей при разработке
ставит доступность использования для широкого круга людей, не
знакомых с устройством компьютеров, операционных систем.

Дополнительно, каждый язык программирования оптимизирован для своего типа
задач, которые предположительно будут создаваться с его использованием.

Постановка задачи создания идеального языка программирования включает в себя
точное и однозначное выражение спецификации этой задачи и способность превратить
написанное в машинные инструкции. Очевидно, что такая цель едва ли достижима,
так как трудно довести до совершенства формулировки большого количества задач,
и еще сложнее реализовать нужный алгоритм.

На данный момент в мире насчитывается более сотни языков программирования,
хотя большинство людей знакомы только с несколькими десятками.

Такое большое количество языков программирования необходимо для качественной
реализации определенного класса приложений. При этом множество современных
языков программирования создаются машинно-независимыми, их структура
не привязана к архитектуре. Отсюда их название, - машинно-независимые языки
программирования высокого уровня.

Сфера компьютерных наук развивается довольно быстро, динамично и хаотично
по причине своей популярности и обширности. Это значит, что постоянно
на рынке появляются новые подходы, старые становятся либо неактуальными, либо
дорабатываются, чтобы соответствовать современным стандартам.

По мере появления более сложных систем и программного обеспечения возникают
всё новые сценарии использования, которые требуют всё новых языков. Сложность
приложений и систем растет, что требует новых методов решения и подходов. При
этом не менее важным параметром является поддержание оптимальной стоимости
разработки.

Как было сказано ранее, на сегодняшний день на первый план вышли языки
программирования высокого уровня, которые позволяют разработчику
сфокусироваться на логике работы приложения и не переживать об
управлении памятью и иными ресурсами, относящимися к низкоуровневым проблемам.
Все это позволяет сформулировать несколько причин для создания
новых языков программирования высокого уровня:

\begin{enumerate}
    \item Проектирование новых языков с целью оптимизации производительности
    может включать в качестве результата более эффективное использование
    ресурсов компьютера, упрощение синтаксиса в контексте целевой задачи
    или разработка новых синтаксических конструкций или функционала,
    обелегчающих работу разработчиков конкретной сферы.
    \item  Проектирование новых языков под новые задачи, для которых на данный
    момент нет удобного или оптимального решения. К таким задачам можно отнести,
    например, задачи, связанные с узко направленной научной деятельностью.
    \item Разработка нового языка может являться чисто стратегическим действием
    для продвижения коммерческих решений в открытый доступ или ради развития
    новых идей, видений и подходов в области компьютерных наук
\end{enumerate}

В настоящее время активно развивается язык программирования, представляющий собой
более быструю альтернативу существующему $TypeScript$ (далее $TS$). Основной целью этого языка
является создание расширенной и более быстрой статической версией $TS$ при
условии сохранения совместимости с самим $TS$ ради упрощения перехода будущих разработчиков
с одного языка на другой.

Современные среды программирования требуют высокую производительность приложений. В таком
контексте принципиально важны дополнительные инструменты, позволяющие автоматически или с
определенной персонализацией под конкретную задачу проанализировать исходный код и выдвинуть
предложения по его оптимизации.

Под повышением производительности в целом могут стоять разные параметры, например,
энергопотребление или размер полученного бинарного файла. Эти и другие параметры обычно
тесно взаимосвязаны, и потому в попытках улучшить показатели одного параметра можно получить
отрицательный результат для другого.

Выделяют динамические параметры программы. К ним традиционно относят производительность
и энергопотребление. Смысл динамических параметров заключается в зависимости от конкретного устройства,
на котором исполняется программа, то есть при проведении замеров результаты не будут повторяться.
Однако такие параметры можно измерять группами: для одной группы устройств и для другой.
Замеры формата 'до' и 'после' в рамках выбранной группы должны отвечать свойству повторяемости.
Дополнительно можно сказать, что тенденция должна сохраняться при переходе из одной группы в другую.
Например, если использование некого алгоритма А в группе А приводит к ускорению приложения на 20\% по
сравнению с методом Б в рамках группы А, то при в группе Б не должно наблюдаться противоположного результата,
то есть, например, замедления на 15\% при таком же сравнении метода А и метода Б. Если такие результаты и
удастся получить, это может означать только некорректность постановки задачи или использованных метрик.

Очевидно, что скорость выполнения программы и энергопотребление являются двумя очень важными
аспектами, при разработке всегда хотелось бы выиграть в обоих. Тем не менее они могут
быть очень по-разному оптимизированы. Тогда некоторые изменения в языке программирования или
в реализации его компилятора могут привести одновременно к увеличению потребления энергии из-за
увеличения числа операций или увеличения нагрузки на процессор, чем мы
не можем пренебрегать в случае устройств с ограниченными ресурсами, и к увеличению скорости выполнения
программы за счет более эффективного распределения ресурсов машины. Некоторые из таких
изменений могут быть следующими:

\begin{enumerate}
    \item При использовании параллелизма алгоритмы увеличивают свою скорость выполнения, но при этом
    могут потреблять большее количество энергии, так как нагрузка на процессор становится
    значительно больше.
    \item При изменении алгоритма или структуры данных некоторые операции могут оказаться
    более эффективными по времени, но менее эффективными по памяти, и тогда расход
    энергии увеличится.
    \item При оптимизациях компилятора запускаются сложные оптимизационные процессы, которые
    могут больше расходовать ресурсы. Это частая проблема при нерациональном или
    неграмотном подходе к оптимизациям.
\end{enumerate}

В данной работе фокус идет на повышение производительности в контексте времени выполнения и
запуска программы за счет гибкой работы со спецификацией целевого языка программирования и
реализацией его компилятора. По данной причине вышеупомянутые оптимизации для работы не являются
релевантными, а использование других, более высокоуровневых абстракций предполагается заранее считать
вредными для быстродействия. Таким образом, будет предлагаться поиск 'недорогостоящих' аналогов
с точки зрения скорости выполнения и спецификации выбранного языка программирования.

Итак, в данном случае уместно считать, что меньшее время исполнения программы
влечет к меньшему энергопотреблению. Далее под повышением производительности будет пониматься скорость выполнения
программы и время ее запуска в рамках выбранных устройств и опций компилятора.



\chapter{Постановка проблемы}

\section{Цели работы:}

\begin{enumerate}
    \item Разработка системы анализа и предупреждений исходного кода выбранного языка программирования высокого уровня 
для повышения его производительности и ускорения запуска написанных на нем приложений.
    \item Реализация системы-аналога $Clang-Tidy$ для выборочного отключения выбранных проверок точечно, непосредственно в исходном коде.
\end{enumerate}

\section{Задачи работы:}

\begin{enumerate}
    \item Изученить существующие решения.
    \item Проанализировать текущую спецификацию целевого языка программирования на предмет потенциально медленных языковых конструкций и функционала,
    сбор данных для дальнейшего тестирования.
    \item Предоставить для каждой языковой единицы среди предложенных вариант исправления, ускоряющий работу приложения, корректный с точки
    зрения выбранного языка программирования.
    \item Протестировать каждое предложение: замерить скорость работы приложения до и после предложенных исправлений.
    \item Разработать систему анализа кода на этапе компиляции.
    \item Реализовать соответствующую поверку в системе анализа после подтверждения положительных результатов тестирования.
    \item Разработать систему точечного и группового отключения выбранных разработчиком проверок, поддержать наиболее популярные сценарии 
    использования, опираясь на данные, полученные при изучении существующих решений.
    \item Поддержать возможность анализа в системе многофайловой сборки.

\end{enumerate}

Цели работы разумно считать достигнутыми при выявлении и реализации не менее пяти предложений,
ускоряющих работу приложения в среднем не менее, чем на 5\%,
а также успешной разработке системы отключения проверок, поддержке проектной сборки и прохождении тестирования, составленного из некоторых
потенциальных сценариев применения предложенных решений.

\chapter{Обзор существующих решений}
Сегодня представлено огромное число средств, позволяющих на разных этапах проанализировать код программы на качество,
причем под качеством могут подразумевать быстродействие, а могут и нет. По данной причине есть смысл
рассмотреть основные группы данных инструментов. Это позволит правильно классифицировать и применить результат данной
работы на практике другим пользователям и читателям.

Итак, разные системы могут включать как статический, так и динамический анализ, и они
используются для различных языков программирования, таких как $C$, $С++$, $Java$, $Python$ и другие.
Для анализа и оптимизации кода существует несколько методов:

\begin{enumerate}
\item \textbf{Статический анализ}

При статическом анализе код подвергается проверке без стадии выполнения.

Примерами инструментов для данного метода могут служить $Lint$ для $C$ или $SonarQube$.

При статическом анализе идет проверка стиля написания кода, поиск потенциальных ошибок.

\item \textbf{Динамический анализ}

При статическом анализе код подвергается проверке во время его выполнения.

Примерами инструментов являются Valgrind для $C$/$С++$, $JProfiler$ для $Java$.

Здесь выполняется профилирование времени выполнения, анализ используемых ресурсов, в том числе
поиск потенциальных утечек в памяти.

\item \textbf{Оптимизации компиляции}
    \begin{enumerate}
        \item \textbf{JIT-компиляция ($Just$ $In$ $Time$)}

        Целью такого рода оптимизаций стоит компиляция кода во время его выполнения, происходит
        оптимизация под конкретное окружение.

        Типичным примером оптимизаций компиляции является $Java$ Virtual Machine (JVM) \cite{jvm-spec}.

        Преимущество данного метода состоит в адаптации кода к конкретным условиям выполнения \cite{jvm-spec}.

        \item \textbf{AOT-компиляция (Ahead-Of-Time)}

        Еще один вид оптимизаций компилятора, но здесь компиляция происходит еще раньше.

        Примером в $С/С++$ является Low Level Virtual Machine ($LLVM$) \cite{clang-tidy}.

        Метод отличается быстрой загрузкой и запуском программ, что помогает уменьшить накладные
        расходы на этапе $JIT$-компиляции.
    \end{enumerate}

\item \textbf{Оптимизация кода}
    \begin{enumerate}
        \item \textbf{Инлайн-функции}

        Целью является уменьшение накладных расходов на вызов функций.
        При инлайне предполагается, что компилятор заменит каждый вызов этой функции
        кодом самого тела этой функции.

        Вызов функции требует поэтапного обращения к стеку, что устраняется путем
        встраивания функции.

        Стоит помнить, что общий размер программы может увеличиться при большом
        количестве инлайн-функций.

        \item \textbf{Удаление мертвого кода}

        Мертвым кодом являются неиспользуемые и недостижимые части программы, что
         способствует уменьшению размера программы, более
        удобному чтению.

        Мертвый код можно обнаружить методами статического анализа.
    \end{enumerate}
\end{enumerate}

Основное, что объединяет вышеперечисленные методы в рамках данной работы, это ограниченность их
анализа спецификацией языка. То есть анализ идет на основе того, как конкретный пользователь
языка использует его конструкции в своей программе, не затрагивая идеи, на которых строился
и писался сам язык программирования.

Теперь рассмотрим инструменты, по сути своей выделяющие из конкретных языков программирования
некоторые подмножества с фокусом на производительность, безопасность или надежность кода.

\begin{enumerate}
        \item \textbf{Ada}

        Язык $Ada$ имеет концепцию ограничительных прагм, которые позволяют разработчикам регулировать
        использование определенного функционала языка ради повышения производительности и надежности
        кода. Такие ограничения выдвигают запрет на использование динамической памяти или
        запрет на использование исключений.

        \item \textbf{SPARK (подмножество Ada)}

        У языка программирования $Ada$ есть подмножество $SPARK$, которое предназначено для
        использования в системах с высокими требованиями к надежности и безопасности \cite{spark}. Данное
        подмножество аналогично прошлому исключает некоторые синтаксические наборы, чтобы
        обеспечить возможность формальной верификации кода, тем самым значительно повышая
        безопасность и надежность написанного.

        \item \textbf{Ada Ravenscar}

        Рассматривая другое подмножество $Ada$, названное $Ravenscar$ \cite{ravenscar}, стоит упомянуть о принципиальной
        разнице со $SPARK$ подмножеством: в данном уже не говорится про обеспечение формальной верификации
        кода. Тем не менее концепция ограничения некоторых возможностей языка сохраняется, чтобы
        обеспечить однозначность выполнения программы.

        \item \textbf{Misra C/С++}

        $MISRA$ (Motor Industry Software Reliability Association) $C$ и $С++$ — это набор
        рекомендаций и правил для написания безопасного и надежного кода на
        языках C и $С++$ \cite{misra}. Эти правила часто используются в автомобильной и
        других критически важных отраслях, где важна высокая производительность и безопасность.

        \item \textbf{CERT C}

        $CERT C$ — это набор руководящих принципов для написания безопасного и надежного
        кода на $C$, разработанный Software Engineering Institute (SEI) при Carnegie Mellon University.
        Стандарт фокусируется на безопасности \cite{cert}: включает рекомендации по предотвращению
        распространенных уязвимостей, таких как переполнение буфера, ошибки с указателями.

        \item \textbf{Standard ECMA-327}

        Данная версия сосредотачивается на минимизации использования памяти и
        процессорного времени для устройств с ограниченными ресурсами \cite{ecma-327}.

        \item \textbf{Java ME (Micro Edition)}

        Для $Java$ существует подмножество под названием $Java ME$ (Micro Edition),
        специально разработанное для устройств с ограниченными ресурсами, таких
        как встроенные системы и мобильные устройства \cite{java-me}. $Java ME$ предоставляет
        сокращенный набор библиотек и API, подходящих для встраиваемых сред,
        и исключает более ресурсоёмкие функции, типичные для $Java SE$ (Standard Edition) \cite{java-se}.
\end{enumerate}

Данные варианты максимально близки идейно с тем, что будет далее представлено в работе.
Теперь, когда проведен обзор разного рода решений, направленных на анализ с целью ускорения
работы программы, ясны методы и инструменты, необходимые для реализации поставленных целей.


\chapter{Теоретическая часть}

Рассмотрим некоторые теоретические аспекты, которые требуются для
выполнения задач данной работы.

\section{Целевой язык программирования}

Для успешной реализации системы анализа кода для повышения производительности программ на
языке программирования высокого уровня требуются некоторые знания о целевом языке программирования.

Выбранный язык сочетает и поддерживает функции, которые используются во многих известных
языках программирования, где эти инструменты уже доказали свою полезность и мощность.
Он поддерживает императивные, объектно-ориентированные, функциональные и обобщенные
 парадигмы программирования, объединяя их безопасно и последовательно.
В то же время целевой ЯП не поддерживает функции, позволяющие разработчикам
программного обеспечения писать опасный, небезопасный или неэффективный код.
В частности, язык использует принцип строгой статической типизации. Он не допускает
динамических изменений типов, так как типы объектов определяются их объявлениями.
Их семантическая корректность проверяется на этапе компиляции.

Основные аспекты, характеризующие данный язык в целом:

\begin{enumerate}
    \item Объектная ориентированность

    Поддержка традиционного подхода к программированию на основе классов и
    объектно-ориентированного программирования (ООП). Основные понятия этого подхода следующие:
    \begin{enumerate}
        \item Классы с единичным наследованием.
        \item Интерфейсы как абстракции, которые реализуются классами.
        \item Виртуальные функции с механизмом динамического переопределения.
    \end{enumerate}
    Объектная ориентированность, общая для многих современных языков программирования,
    обеспечивает мощный, гибкий, безопасный и понятный дизайн программного обеспечения.

    \item Модульность

    Язык поддерживает компонентный подход к программированию. Предполагается, что
    программное обеспечение разрабатывается и реализуется как композиция единиц
     компиляции. Единица компиляции обычно представлена в виде модуля или пакета.

    \item Статическая типизация

    Связь с типом, например, возвращаемого значения функции или переменной, происходит сразу, в момент объявления,
    и не может быть изменена позже \cite{eng-article}.

    \item Управляемость

    Язык работает в управляемой среде выполнения, и управление ресурсами осуществляется не напрямую
    операционной системой, а через промежуточное программное обеспечение \cite{eng-article}, которое обеспечивает ряд услуг,
    таких как управление памятью и сборка мусора (garbage collection).
\end{enumerate}
\section{Бенчмаркинг}

После исследования спецификации целевого языка программирования требуется для каждого
потенциально медленного метода А, имеющего альтернативу в виде метода Б, провести
замеры и сравнить скорость выполнения кода или время старта выполнения. Для выполнения
данной задачи в работе используется подход, называемый бенчмаркингом.

В общем случае бенчмарком называется программа, которая измеряет некоторые характеристики
производительности приложения или фрагментов кода \cite{benchmark-book}. В данной работе предлагается считать
бенчмарк экспериментом, так как с его помощью должны получаться результаты, позволяющие
узнать более подробно поведение приложения.

Получив значения метрик, нужно объяснить их и быть уверенным в том, что предложенное
объяснение является верным. В ходе работы часто будет использоваться бенчмарк для
сравнения некого метода А и метода Б в рамках конкретного функционала языка.

Тривиальное сравнение скорости методов с последующим предпочтением
более быстрого не является
корректным решением ни одной из задач данной работы, так как является в корне неверным подходом.
Подобного рода вывод носит строго локальный характер, исключает анализ данных, на которых
проводились замеры, состояние системы, причины увиденной разницы в скорости \cite{benchmark-book}.

Зачастую аномальные значения в результатах замеров производительности связаны с
ошибками в методологии измерений. Таким образом, фиксация прироста скорости метода А по сравнению
с методом Б не является решенной задачей. Кроме того, бенчмаркинг в целом не является
универсальным подходом, полезным при любом исследовании производительности.

Одним из наиважнейших требований к качественному бенчмаркингу является повторяемость
полученных результатов. Предполагается, что между измерениями допустима только незначительная
разница, не влияющая на выводы из эксперимента и не дающая качественных различий в полученных
результатах, относящихся к разным запускам.

\section{Абстрактное синтаксическое дерево (AST)}

При реализации системы анализа возник вопрос о том, как именно его производить, чтобы это
было максимально быстро и корректно. Выбор был принят в пользу изучения абстрактного
синтаксического дерева (Abstract Syntax Tree или AST) после прохода основного
этапа проверки, заключающей, сожержит ли программа ошибки времени компиляции. Соответственно, необходимо
уметь понимать это дерево на выходе.

Итак, у каждого ЯП есть набор ключевых слов, по которым мы определяем некоторые выражения в программе,
задаем инструкции и получаем из наших входных данных определенный результат. AST отличается понятной
структурой для человека.

Можно определить язык программирования высокого уровня как язык с упором на читаемость человеком. Но
программа на языке высокого уровня не запустится, если не перевести ее в код на языке низкого уровня. За
данный этап ответственны два процесса: компиляция и интерпретация.

Разработать код на языке высокого уровня и перевести его в список команд низкого уровня позволяют AST-деревья.
Абстрактное синтаксическое дерево представляет собой один из промежуточных слоев при преобразовании языков
высокого уровня.

Устройство AST следующее \cite{dragon-book}:

\begin{enumerate}
    \item Промежуточными узлами являются функции, методы, операторы.
    \item В листах дерева содержатся константы, переменные, методы классов, аргументы.
\end{enumerate}

AST-дерево не является деревом разбора, так как не учитывает синтаксис языка.

\section{Спецификация языка программирования}

Cпецификация языка программирования — это документ, который определяет язык программирования, 
чтобы пользователи и разработчики могли однозначно понимать,
что означают программы на этом языке \cite{tsspec} \cite{spec-language}.

Спецификация языка программирования может принимать несколько форм, включая следующие \cite{sun-test}:

\begin{enumerate}
    \item Явное определение синтаксиса и семантики языка. В целом, синтаксис обычно 
    задается с использованием формальной грамматики, семантические определения 
    могут быть написаны на естественном языке (например, подход, использованный 
    для языка C) или формальной семантики (например, спецификации Standard ML и Scheme).
    \item Описание поведения транслятора языка (например, $С++$ и $Fortran$).
    Синтаксис и семантика языка должны быть выведены из этого описания так, 
    чтобы его можно было написать на естественном или формальном языке.
    \item Модельная реализация, иногда написанная на целевом языке. Модельная реализация
    является по определению стандартом, на основе которой измеряются все другие реализации с 
    соответствующими начальными условиями и к которой добавляются все улучшения.Синтаксис 
    и семантика языка явны в поведении модельной реализации. 
\end{enumerate}

Синтаксис языка программирования обычно описывается, используя комбинацию следующих двух компонент \cite{tsspec}:
\begin{enumerate}
    \item Регулярные выражения для описания лексем.
    \item Грамматики, которые описывают, какие именно комбинации лексем
    корректны синтаксически.
\end{enumerate}

Формулирование строгой семантики большого, сложного и применимого на практике языка 
программирования является непростой задачей даже для опытных специалистов, 
а полученная спецификация может быть трудной для понимания пользователем.
Рассмотрим некоторые способы описания семантики языка 
программирования:
\begin{enumerate}
    \item Естественный язык: описание на естественном языке.
    \item Формальная семантика: математическое описание.
    \item Эталонные реализации: описание с помощью кода программы.
    \item Тестовые наборы: описание с использованием примеров программ
    и их ожидаемого поведения.
\end{enumerate}

Стоит отметить, что все языки используют хотя бы один из этих методов 
описания, а некоторые сочетают в себе сразу несколько.

Наиболее широко используемые языки определяются с использованием описаний 
их семантики на естественном языке. Это описание обычно принимает форму 
справочного руководства по языку. Причем такие документы могут содержать сотни 
страниц, например, печатная версия
«Спецификации языка $Java$», 3-е издание имеет объем 596 страниц \cite{js-spec}.

Неточность естественного языка как средства описания семантики 
языка программирования может привести к проблемам с интерпретацией 
спецификации. Например, семантика потоков в $Java$ была определена на
английском языке, и позже выяснилось, что спецификация не дает 
качественных указаний для разработчиков \cite{jvm-spec}.

\section{$Clang-Tidy$}

$Clang-Tidy$ — это инструмент, представляющий собой «линтер» $С++$ на основе clang.
Его цель — предоставить расширяемую среду для диагностики и 
исправления типичных ошибок 
программирования, таких как нарушения стиля, неправильное использование 
интерфейса или поиск ошибок с применением статического 
анализа \cite{clang-tidy}. $Сlang-Tidy$ является модульным и предоставляет удобный
интерфейс для написания новых проверок.

$Сlang-Tidy$ имеет свои собственные проверки, может запускать
проверки Clang Static Analyser. Каждая проверка имеет имя, 
а сам список проверок можно выбрать с помощью специального параметра $-checks=$ \cite{clang-tidy}.

Однако опция $-checks=$ не влияет на аргументы компиляции, поэтому она 
не может включать предупреждения $Clang$, которые еще не включены в
конфигурации сборки. Параметр $-warnings-as-errors=$ преобразует 
любые предупреждения, выдаваемые под флагом $-checks=$, в ошибки,
но сам по себе никаких проверок не включает.

Диагностика $Clang-Tidy$ предназначена для вызова кода, который не
соответствует стандарту кодирования или каким-либо образом проблематичен. 
Однако, если известно, что код правильный, может быть полезно отключить предупреждение.

Если конкретный механизм подавления недоступен для определенного 
предупреждения или его использование по какой-либо причине 
нежелательно, $Сlang-Tidy$ имеет общий механизм подавления 
диагностики через комментарии формата $NOLINT$,
$NOLINTNEXTLINE$ и $NOLINTBEGIN… NOLINTEND$.

Комментарий $NOLINT$ предписывает $Clang-Tidy$ игнорировать предупреждения
в той же строке, то есть он не применяется к функции, блоку кода или любой 
другой языковой конструкции.
Комментарии $NOLINTBEGIN$ и $NOLINTEND$ позволяют подавлять предупреждения 
$Clang-Tidy$ в нескольких строках, расположенных между двумя
комментариями, включающих, соответственно, $NOLINTBEGIN$ и $NOLINTEND$.

За всеми комментариями может следовать необязательный список имен 
проверок в круглых скобках. 
Список имен проверок поддерживает подстановку с тем же форматом 
и семантикой, что и при включении проверок.

Формальный синтаксис имеет следующее описание:
\begin{verbatim}
            lint-comment:
              lint-command
              lint-command lint-args

            lint-args:
              ( check-name-list )

            check-name-list:
              check-name
              check-name-list , check-name

            lint-command:
              NOLINT
              NOLINTNEXTLINE
              NOLINTBEGIN
              NOLINTEND
\end{verbatim}

Конкретный пример:

\begin{verbatim}
            class Foo {
              // Отключение диагностики для данной линии
              Foo(int param); // NOLINT

              // Отключение конкретной проверки со следующей линии
              // NOLINTNEXTLINE(google-explicit-constructor)
              Foo(bool param);

              // Отключение всех проверок из google модуля
              // для всех линий между BEGIN и END
              // NOLINTBEGIN(google*)
              Foo(bool param);
              // NOLINTEND(google*)
            };
\end{verbatim}

Все рассмотренные выше разделы в комбинации будут применяться для выполнения задач 
данной работы.

\chapter{Практическая часть}

Изучив существующие решения, перейдем непосредственно к анализу текущей спецификации
целевого языка программирования на предмет потенциально медленных языковых конструкций и функционала и
сбору данных для дальнейшего тестирования.

Недостаток данной работы заключается в невозможности полного анализа спецификации, так как 
выбранный ЯП находится в стадии активной разработки. Несколько раз в месяц документ обновляется, и одни 
положения из него уходят, одновременно с тем как приходит много новых глав и подразделов, исправляются 
ошибки.

По данной причине на данном этапе была выбрана стратегия исследования двух категорий 
подмножеств языка:

\begin{enumerate}
    \item Подмножество, общее с $TypeScript$. То есть та часть языка, которая может быть аналогично
    воспроизведена с помощью $TypeScript$, без внесения каких-либо изменений в текущий код программы.
    \item 'Фундаментальное' подмножество. Будем условно считать в рамках данной работы фундаментальными
    те конструкции языка, при изменении которых потребуются значительные изменения в
    комплементарных главах спецификации, и значительно меняющие поведение 
    программы на целевом языке программирования.
\end{enumerate}

Конечно, спецификация ЯП не делится на описание подмножеств подобного характера. Поэтому их поиск 
заключает в себе прежде всего последовательное чтение документа с претензией на его неоптимальность.
В текущих условиях то, какие именно гипотезы пошли на дальнейшее тестирование, зависит
исключительно от опыта и знаний автора работы.

\section{Работа со спецификацией}

\subsection{Имплицитный боксинг и анбоксинг}

Одна из первых глав спецификации содержит описание типов в данном языке
программирования. Упоминается о том, что выбрана статическая типизация, что, как 
было упомянуто в соответствующем теоретическом разделе, означает, что тип 
каждой объявленной сущности и каждого выражения известен в
время компиляции. Тип объекта либо явно задается разработчиком, 
либо неявно выводится компилятором.

Существует две категории типов:
\begin{enumerate}
    \item Типы значений (Value Types).
    \item Ссылочные типы (Reference Types).
\end{enumerate}

Типы, встроенные в целевой ЯП, называются предопределенными.

К предопределенным типам относятся примитивные типы данных, в том числе 
$byte$, $short$, $int$, $long$, $float$, $double$, $boolean$, $char$.

Каждому предопределенному типу значения соответствует 
предопределенный тип класса, называемый 
упаковочным (или боксингом), который обертывает значение предопределенного 
Value Type: $Number$, $Byte$, $Short$, $Int$, $Long$, $Float$, $Double$, $Boolean$, $Char$.

Перевод из Value Type в соответствующий Reference Type называется упаковкой
(боксингом). Обратный перевод называется распаковкой (анбоксингом).

Спецификация языка программирования позволяет для обеих категорий типов выполнять ряд
операций, например, арифметических. То есть корректно с точки зрения языка
объявление двух переменных Value Type $int$ и их последующее сложение с записью
результата в переменную такого же типа наравне с аналогичной операцией для Reference Type $Int$.

Справедливо предположение о том, что для выполнения примитивных операций и ряда других
сценариев необязательно использование более 'сложных' типов данных, которые несут 
в себе дополнительный функционал.
Для подобных сценариев отлично подходят более 'легкие' Value Types, так как для работы 
с ссылочным типом надо аллоцировать место в куче. Значимый тип может работать на 
стеке, не уходя в кучу, а может стать частью ссылочного типа. 
Это свойство может значительно повысить производительность для некоторых программ.
Дополнительно, при многих, казалось бы, обычных операциях, происходит постоянная неявная
конвертация из Value Type в Reference Type, что просто по определению несет в себе 
дополнительные затраты ресурсов, увеличивая время исполнения программы.

Итак, предложена проверка на выявление неявной упаковки и распаковки с предложением
переписывания на соответствующие упрощающие типа данных в каждом контексте.


\subsection{Модификатор $final$ для методов и классов}

Некоторая вводная информация про классы в выбранном ЯП: объявления классов 
вводят новые ссылочные типы и описывают способ их реализации;
классы могут быть верхнего уровня (Top-level) или локальными;
тело класса содержит объявления и инициализаторы классов;
объявления могут представлять члены или конструкторы классов;
тело объявления члена включает в себя область действия объявления;
члены класса включают в себя поля, методы и аксессоры; члены класса могут 
быть объявлены или унаследованы.

Что принципиально важно, по умолчанию все классы в целевом ЯП являются открытыми
для наследования. То есть при отсутствии модификатора $final$ у метода или 
у самого класса, можно производить наследование.

Именно с этого момента возникает проблема. Язык мог бы иметь противоположный
модификатор по умолчанию, как это сделано в $Kotlin$, однако это подразумевает большой пласт несовместимости
с $TypeScript$. Рассмотрим следующий частый пример: разработчик пишет свою программу, 
определяет классы, интерфейсы и не продумывает детально механизмы взаимодействия между ними.
Соответственно, в итоге программа содержит много классов, которые не имеют никаких 
наследников, как и их методы, и данная программа даже отлично работает и не выдает
никак проблем.

Однако очевидна неоптимальность: классы  и методы, не имеющие наследников,
не помечены модификатором $final$. При вызове метода $TypeScript$ проверяет,
есть ли этот метод
в самом классе \cite{tsspec}. Если метод не найден, он ищет его в родительском классе,
продолжая вверх по цепочке наследования до тех пор, пока не найдет метод
или не достигнет верхнего уровня (базового) класса. Вставка $final$ позволяет
не делать лишние проверки и поиски в случае отсутствия наследников.


\subsection{Инструкции верхнего уровня}

Инструкции верхнего уровня (Top-level statements) - это языковые конструкции, допустимые
к использованию без необходимости оборачивать их в функции, классы или иные структуры
данных. Отдельный модуль может содержать последовательности операторов,
которые логически составляют одну единую последовательность операторов.

Если один модуль импортирует другой, то семантика операторов
состоит в определении импортируемого модуля.
Таким образом, все операторы верхнего уровня выполняются один раз до
вызова какой-либо другой функции или до доступа к какой-либо
переменной верхнего уровня.

Если отдельный модуль используется в качестве программы, то инструкции
верхнего уровня используются как точка входа программы. Если отдельный
модуль содержит $main$ функцию, то она выполняется сразу после
исполнения инструкций верхнего уровня.

Ошибка времени компиляции (Compile Time Error) в целевом ЯП возникает, если
инструкции верхнего уровня содержат $return$.

Инструкции верхнего уровня можно использовать
непосредственно в теле файла без необходимости оборачивать их в функции,
классы или другие структуры. Эта возможность существует во многих ЯП, например,
в $C\#$ и $Kotlin$. Это позволяет исполнять код сразу при запуске программы, что делает её
удобной для написания простых скриптов и небольших программ.

Инструкции верхнего уровня просты в использовании и потому очень удобны
для небольших приложений и скриптов, где не требуется большой работы
с ресурсами. Кроме того, линейность этих и инструкций делает код
более простым для понимания.

Выскажем предположение, что это же порождает проблемы в более
высоко нагруженных системах. Например, если выполняются какие-либо
тяжелые вычисления или операции ввода-вывода, это уже значительно
увеличивает время запуска приложения, что может уже быть критично
при разработке в мобильной сфере, так как пользователи не захотят
долго ждать. Кроме того, второй популярный сценарий разработки -
использование библиотек и фреймворков, при их подгрузке время
запуска растет значительно.

Перемещение инструкций верхнего уровня
в ту же функцию $main$ избавляет от потенциально лишних операций
и увеличивает вероятность быть оптимизированным за счет
($JIT$) \cite{jvm-spec} компиляции.
Например, если модуль импортируется, но не вся импортированная
функциональность используется, то инструкции верхнего уровня все
равно выполнятся полностью и засорят код. В случае перемещения в
функцию $main$ гарантируется, что импорты будут только
при необходимости.

\subsection{Операторы равенства}

Согласно спецификации выбранного ЯП, выражения равенства
используют операторы равенства $==$, $===$, $!=$ и $!==$.
Во всех случаях выражение $a != b$ дает тот же результат,
что и $!(a == b)$, а выражение $a !== b$ дает тот же результат, что и $!(a === b)$.

Рассмотрим случай конкретно ссылочного типа данных. Сравнение ссылок применяется,
если оба операнда являются совместимыми ссылочными типами,
за исключением типов $string$, $bigint$, $Object$, объединенных типов
($Union$) и типовых параметров ($Type Parameters$).

Целевой статически типизированный ЯП предоставляет
расширенную семантику для сравнения с $null$ и $undefined$
ради более полного соответствия $TypeScript$. Любая сущность может
быть сравнена с $null$ с использованием операторов $==$ и $!=$. Это
сравнение может вернуть значение $true$ только для $nullable$ типов,
если они действительно $null$ во время выполнения
программы ($runtime$). В остальных случаях сравнение с $null$ возвращает $false$,
и это известно во время компиляции.

Аналогично, сравнение с $undefined$ есть $false$. Но, если
переменная, с которой производится сравнение, является типа
$undefined$ или объединенного типа, который включает
$undefined$ как один из своих типов - получаем $true$.

При изучении реализации данного языка программирования на основе
спецификации было выдвинуто предположение о том,
что одна и та же проверка может быть выполнена с разной скоростью в
зависимости от порядка расположения операндов в операторе равенства, а
именно $ref == null$ или $null == ref$.

Данный эффект ожидаем в связи с тем, что первым шагом при вычислении
значении такого равенства идет проверка левой части на $null$, а
только потом уже сравнение операндов. Во втором же случае можно
предположить, что проверка на $null$ по сути опускается, что сильно
должно уменьшать время исполнения, ведь подобного рода проверки делаются
сильно чаще и не являются примитивными, так как речь идет о ссылочном
сравнении.

\subsection{Использование корутин вместо асинхронных функций}

Асинхронные функции в $TypeScript$ представляют собой функции,
которые используют ключевое слово $async$. Они позволяют выполнять
операции асинхронно и возвращать $Promise$. Внутри такой функции
могут использоваться операторы $await$, которые приостанавливают
выполнение функции до тех пор, пока не завершится операция,
указанная после $await$.

Корутины же позволяют явно управлять состоянием выполнения \cite{cplusplus}, что может
быть полезно для сложных потоков работы или взаимодействия
с внешними ресурсами. Корутины могут приостанавливать своё
выполнение ($yield$) и возобновлять его позже, сохраняя
состояние между вызовами.

Асинхронная функция является неявной корутиной, которую нельзя
вызывать как обычную функцию. Использование такой аннотации
не рекомендуется данным ЯП. Этот тип функций поддерживается
только для обратной совместимости с $TypeScript$.
Так как данный язык в целом ориентирован на ускорение $TS$, это
добавляет аргумент в пользу необходимости сравнения асинхронных
функций и корутин.

\subsection{Использование вызова функций вместо лямбд}

Лямбда — это короткий блок кода, который принимает параметры
и может возвращать значение \cite{cplusplus} \cite{tsspec}.
Лямбды похожи на функции, но не требуют имен
в описаниях. Лямбды могут быть реализованы непосредственно в выражениях.

Под обычными функциями в данном контексте подразумеваются
функции, объявленные с использованием ключевых слов $function$
и других аналогичных.

При вызове лямбда-функции создаётся объект и определяется
виртуальная функция. При использовании будет
выполнен виртуальный вызов, который требует больше времени,
чем вызов обычной функции. В среднем, при повторном вызове
в лямбда-функции, эти различия должны занимать больше времени.
Поэтому данная гипотеза пригодна для проверки бенчмаркингом
на иследующем этапе.

\subsection{Остаточные параметры (rest parameters)}

Остаточные параметры в $TypeScript$ позволяют функции
принимать переменное количество аргументов в виде массива \cite{tsspec}.
Обозначение таких параметров реализовано с помощью троеточия $(...)$
перед именем параметра.

Таким образом, когда мы используеем $(...)$ перед именем параметра,
$TypeScript$ собирает все переданные аргументы, начиная с этого места, в массив.

Возникает соответствующая мысль о том, что можно сразу положить все переданные
аргументы в массив и не создавать его каждый раз с нуля при остаточных параметрах
в сценарии, например, большого цикла. Тогда массив создастся ровно один раз с проверкой
наличия необязательных параметров. Относительно неплохой альтернативой между
вариантами массива и остаточных параметров могут быть опциональные параметры.

В спецификации выбранного языка программирования говорится о двух формах
опциональных параметров. Первая форма содержит выражение, которое
указывает значение по умолчанию. Значение параметра устанавливается
в значение по умолчанию, если аргумент, соответствующий
этому параметру, не передается при вызове функции.

Вторая форма является краткой записью для параметра объединенного типа
$T | undefined$ с значением по умолчанию $undefined$.

Есть необходимость сравнить поведение массивов, опциональных и остаточных параметров

\section{Бенчмаркинг выдвинутых гипотез}

Итак, когда в рамках просмотра было выдвинуто по меньшей мере 7 идей для ускорения,
необходимо их протестировать. Тестирование производится путем бенчмаркинга в стиле
сравнения варианта А с вариантом Б \cite{benchmark-book}.

Пройдемся по каждой из выдвинутых гипотез.
\subsection{Имплицитный боксинг и анбоксинг}
Имплицитный боксинг и анбоксинг типов случается огромное множество раз в рядовой программе
на том же $TypeScript$. Постоянное преобразование типов в $switch$ и просто при выполнении
примитивных операций, сравнений или передаче аргументов в функцию довольно сильно замедляет
программу, согласно замерам.

При бенчмаркинге стандартной
настройкой являются максимальные оптимизации компилятора.

Ниже приведены результаты бенчмарков нескольких форматов.

\begin{enumerate}
    \item Тип аргументов $double$
\begin{table}[h]
    \centering
    \begin{tabular}{|c|c|c|c|}
      \hline
      Тип запуска & Боксинг аргумента & Аргумент без боксинга & Ускорение \\
      \hline
      Время в сек& 3,750 & 1,750 & 53,3\% \\
      \hline
    \end{tabular}
    \caption{Тип запуска $double$}
  \end{table}

  \item Тип аргументов $int$

\begin{table}[h]
    \centering
    \begin{tabular}{|c|c|c|c|}
      \hline
      Тип запуска & Боксинг аргумента & Аргумент без боксинга & Ускорение \\
      \hline
      Время в сек& 19,776 & 1,148 & 94,2\% \\
      \hline
    \end{tabular}
    \caption{Тип запуска $int$}
  \end{table}

  \item Выполнение в $switch$
  \begin{table}[h]
    \centering
    \begin{tabular}{|c|c|c|c|}
      \hline
      Тип запуска & Боксинг & Без боксинга & Ускорение \\
      \hline
      Время в сек& 2,149 & 1,061 & 50,6\% \\
      \hline
    \end{tabular}
    \caption{Выполнение в $switch$}
  \end{table}
\end{enumerate}

По данным таблицы очевиден большой прирост в скорости в отсутствии
лишних перегонок из упакованного типа в неупакованный и наоборот.
Таким образом, проверка на неявную упаковку и распаковку должна
присутствовать в системе анализа и предупреждений.

\subsection{Модификатор $final$ для методов и классов}

Согласно выдвинутой гипотезе, предложение использовать модификатор
$final$ для классов и методов, не имеющих наследников, должно существенно
ускорить выполнение программы, так как больше нет необходимости в поисках
по таблице методов при вызове.

Наследники ожидаются в том числе в качестве локальных классов. Одновременно с этим,
нельзя предлагать использование данного модификатора в местах, где это в соответствии
со спецификацией выбранного языка программирования повлечет ошибку времени компиляции.
То есть во всех случаях вариант А и вариант Б приводят к одинаковому результату выполнения
программы.

Запрещено использование ключевого слова $final$ для абстрактных классов, классов или методов
с наследниками, а также статических.

\begin{table}[h]
    \centering
    \begin{tabular}{|c|c|c|c|}
      \hline
      Тип запуска & Без $final$ & С $final$ & Ускорение \\
      \hline
      Время в сек & 25,107  & 21,800 & 13,1\% \\
      \hline
    \end{tabular}
    \caption{Модификатор в $final$}
  \end{table}

Итоговый прирост составляет
порядка 13\%, что является отличным результатом для легковесного бенчмарка.

Очевидно, что на более крупных системах итоговый прирост будет в процентном соотношении не
таким заметным. Однако много маленьких ускорений влекут хороший суммарный результат.

Таким образом, данный тип проверки тоже рекомендован к включению в анализирующую систему.

\subsection{Инструкции верхнего уровня}

Запрет на использование инструкций верхнего уровня предполагает ускорение
времени запуска приложения. То есть время, которое требуется затратить, чтобы
приступить к исполнению $main$ функции.

В распоряжении автора работы для оценки такого эффекта не имеется полноценных
программных инструментов, чтобы произвести корректную и точную оценки,
поэтому замеры для данной гипотезы были проведены с использованием подручных средств
и имеют большую погрешность в результатах.

Тем не менее прирост очевиден и велик, если в качестве сценария взять, например
вычисление суммы двух чисел в цикле, и вариант А заключить в виде инструкций верхнего
уровня, а в случае варианта Б в виде тела функции $main$. Этот пример легко
масштабируется до уровня библиотек и фреймворков.

Потому, данную категорию запретов стоит внести в анализирующую систему аналогично
паре предыдущих.


\subsection{Операторы равенства}

Обоснование гипотезы в разделе работы со спецификацией дало аналогичные
результаты в виде целевого байт-кода, и гипотеза подтвердилась. Данный результат
абсолютно закономерен, ведь уменьшение количества проверок без уменьшения
безопасности используемого кода является очевидной оптимизацией.

\begin{table}[h]
    \centering
    \begin{tabular}{|c|c|c|c|}
      \hline
      Тип запуска & $ref == null$ & $null == ref$ & Ускорение \\
      \hline
      Время в сек & 5,261 & 4,073 & 22,5\% \\
      \hline
    \end{tabular}
    \caption{Операторы равенства}
  \end{table}

Отдельно отметим, что переписывание
кода с варианта А на вариант Б абсолютно тривиально и не требует никаких дополнительных
исследований на корректность и валидность с точки зрения языка.

Таким образом, проверка
на операторы равенства тоже будет добавлена в анализирующую систему.

\subsection{Использование корутин вместо асинхронных функций}

Использование корутин носит несовместимый с $TypeScript$ характер \cite{tsspec}. Кроме того,
предложение по переписыванию асинхронных функций на корутины имеет больше стилевой
и инженерный смысл к текущему моменту развития спецификации. Это означает, что
большого прироста производительности не ожидается. Однако замеры имеют следующий характер:

\begin{enumerate}
    \item Использование со $string$
    \begin{table}[h]
        \centering
        \begin{tabular}{|c|c|c|c|}
          \hline
          Тип запуска & Асинхронный & Корутины & Ускорение \\
          \hline
          Время в сек & 1,318 & 1,426 & -7,6\% \\
          \hline
        \end{tabular}
        \caption{Использование со $string$}
      \end{table}
    \item Использование с $int$
    \begin{table}[h]
        \centering
        \begin{tabular}{|c|c|c|c|}
          \hline
          Тип запуска & Асинхронный & Корутины & Ускорение \\
          \hline
          Время в сек & 1,915 & 1,886 & 1,5\% \\
          \hline
        \end{tabular}
        \caption{Использование с $int$}
      \end{table}
\end{enumerate}

Видно, что результаты носят не общий характер, и все зависит от использования
конкретных типов. Тем не менее есть случаи, удачные для оптимизации. С учетом
позиции, указанной в спецификации языка как приоритетной с точки зрения стиля и,
возможно, будущего видения развития проекта, данная проверка будет внесена в
анализирующую систему с ограничениями.

\subsection{Использование вызова функции вместо лямбд}

В общем случае, вызов обычных функций ожидается немного
более эффективным, чем вызов лямбд. Это ожидание базируется на
дополнительной работе, которую компилятор должен выполнить для обработки лямбд.

Выдвинутая гипотеза подтверждается в замерах бенчмаркингом:

\begin{table}[h]
    \centering
    \begin{tabular}{|c|c|c|c|}
      \hline
      Тип запуска & Лямбды & Функции & Ускорение \\
      \hline
      Время в сек & 11,643 & 9,479  & 18,5\% \\
      \hline
    \end{tabular}
    \caption{Функции и лямбды}
  \end{table}

Данный бенчмарк отмечает существенный прирост скорости между вариантом
А и вариантом Б. Данный тип проверок имеет смысл включить
в предложенную в теме работы систему.

\subsection{Остаточные параметры (rest parameters)}

Есть 3 способа передачи неопределенного количества параметров функции:
остаточные параметры, опциональные параметры и массивы, поэтому сравним их.

Результаты имеют характер, представленный в таблице \ref{tab:example}.

\begin{table}[h]
    \centering
    \begin{tabularx}{\textwidth}{|c|c|c|c|X|}
      \hline
      № & $Rest$, c & Массив, c & Опциональные, c & Комментарии \\
      \hline
      1 & 10331 & 556 & 6768 & Массив строится 1 раз до цикла с проверкой неиспользуемых параметров \\
      \hline
      2 & 10319 & 10325 & 6715 & Массив строится на каждой итерации с аналогичной проверкой\\
      \hline
      3 & 10286 & 10410 & 6682 & Массив строится на каждой итерации без проверки неиспользуемых параметров\\
      \hline
    \end{tabularx}
    \caption{Функции и лямбды}
    \label{tab:example}
  \end{table}



Из данной таблицы совершенно отчетливо прослеживается тенденция:
остаточные параметры медленнее в разы, чем массивы или опциональные параметры.
Выбор использования массива или опциональных параметров для переписывания больше
зависит от конкретной задачи.

Остаточные параметры по определению являются синтаксическим сахаром, то есть является
более кратким, интуитивно понятным способом записи с точки зрения человека, не добавляя
новых возможностей или изменений в язык. В данной ситуации это отчетливо прослеживается
наличием двух конструкций, позволяющих получить тот же результат работы программы.

Все эти наблюдения позволяют внести остаточные параметры в список пунктов проверки
будущей системы анализа и предупреждений.

\section{Разработка анализирующей системы}

Переходим к разработке самой системы, выдающей предупреждения по неэффективности кода.

Совершенно бессмысленно выполнять
анализ на эффективность кода до потверждения его корректности.
Поэтому данная система будет идти следующим этапом и работать по тому же принципу, что
проверка на мертвый код. Мертвым кодом называются участки программы,
которые либо не могут быть исполнены, либо их выполнение никак не влияет на
результат работы программы \cite{dragon-book}.

Так как система анализа и предупреждений не является обязательной для
достижения корректности приложений, для реализации предложена
следующая структура анализатора:
\begin{enumerate}
    \item Все проверки активируются или
    деактивируются через механизм флагов компиляции.
    \item Флаги разделены на группы по
    принципу общности с $TypeScript$ и, наоборот, отсутствию совместимости с ним.
    \item Существует флаг, запускающий абсолютно все проверки системы.
    \item По аналогии с предупреждениями компиляторов $gcc$ и $clang$ \cite{clang-tidy},
    есть возможность воспринимать все выдвинутые системой предупреждения
    как ошибки времени компиляции.
\end{enumerate}

Данная структура является понятной и довольно простой в реализации при условии
корректной работы с $AST$.

Процесс работы анализатора на основе предложенной схемы в итоге выглядит следующим образом:

\begin{enumerate}
    \item Происходит считывание флагов компиляции. Если среди них есть флаги, относящиеся к
    системе анализа и предупреждений, она запустится после проверки программы на корректность.
    \item Система анализа при вызове получает программу как набор узлов, имеющих свой тип после
    прохождения этапа компиляции \cite{pierce}. Например, существуют узлы присвающего выражения, узлы
    декларации классов и другие.
    \item В таком виде программа проходит каждую проверку из списка активированных. Обход
    дерева происходит от самого его корня, постепенно спускаясь от родительских узлов к дочерним.
    \item Сами проверки завязаны на имеющийся тип текущего узла и на контекст остального
    поддерева \cite{pierce}, необходимого для однозначного решения о вынесении предупреждения о неэффективности.
    Например, в случае проверки на модификатор $final$ для классов и методов, происходит поиск
    наследников по всей программе, в том числе в других модулях и среди локальных классов.
    \item При выполнении условий, соответствующих возможности оптимизировать код, выносится
    соответствующее сообщение, которое содержит имя проверенного файла, строку и индекс
    проверенного узла и текст, описывающий неэффективность с предложением исправления.
    \item В случае, если активна опция за восприятие предупреждений как ошибок, работа
    анализирующей системы останавливается сразу после первого выданного сообщения, происходит
    ошибка времени компиляции.
\end{enumerate}

Этот процесс работы получится несколько видоизмененным после внедрения запрещающей системы.

\section{Разработка запрещающей системы}

Как было сказано в предыдущем пункте, опции компиляции регулируют, какие проверки будут
активированными для конкретной программы, а какие будут отключенными. Это позволяет хорошо
персонализировать анализатор под нужды разработчика. Тем не менее этого недостаточно.

Популярный сценарий использования аналогичных систем, как описанный ранее $Clang-Tidy$,
позволяет точечно добавлять или отключать проверки в самом коде программы в виде
комментариев специального формата.

Для данной системы внедрен аналогичный функционал: отключение проверок для текущей строки, для
следующей строки, для блока кода, заключенного между комментариями, обозначающими начало и конец
выбранных строк. Реализация выполнена в том же стиле, что и $Clang-Tidy$, то есть в виде
специализированных однострочных и многострочных комментариев \cite{clang-tidy}.

Процесс работы запрещающей системы выглядит следующим образом:
\begin{enumerate}
    \item При триггере анализирующей системы происходит парсинг кода на предмет наличия специальных
    комментариев запрещающей системы.
    \item Все узлы, имеющие запрет, помечаются специальным флагом типа, связанного с запретом конкретной проверки.
    \item При работе анализирующей системы все узлы перед началом анализа проверяются на наличие запрещающего флага типа,
    относящегося к данной категории проверки. То есть, если есть узел, помеченный запретом только на проверку неявной упаковки,
    запрет сработает при проверке этого узла на неявную упаковку, но никакого запрета не будет при анализе данного узла на,
    например, остаточные параметры.
    \item При отсутствии запрета узел будет проверяться, а при его присутствии будет просто пропущен анализирующей системой
    для анализа, но не будет удален из анализирующего поддерева. Также рекурсивный спуск от корня к дочерним узлам не
    останавливается при встрече запрета, что позволяет не упускать связанные блоки из анализатора.
\end{enumerate}

Таким образом, разработана запрещающая система для данного анализатора и сам анализатор. Для достижения целей работы
осталось выполнить одну задачу, связанную с системой проектной сборки.

\section{Поддержка анализа через систему сборки}

Как известно, язык программирования $TypeScript$ в качестве системы сборки использует так называемый
$tsconfig$ \cite{tsspec}, который представляет собой обычный $json$-файл. Этот файл выполняет следующие задачи:

\begin{enumerate}
    \item Установка корневого каталога проекта $TypeScript$.
    \item Установка файлов проекта.
    \item Настройка параметров компиляции.
\end{enumerate}

Для его использования нужно вручную добавить новый файл с именем $tsconfig.json$ в корень проекта.


Каждый $tsconfig.json$ представляет собой файл, который содержит ряд секций.
Например, секция $compilerOptions$ настраивает параметры компиляции.
Здесь можно указать необходимые параметры и их значения.
Параметры называются так же, как и в командной строке.
Это значит, что выставить значения можно те же, что передаются в командной строке. Например:

\begin{verbatim}
        {
            "compilerOptions": {
                "target": "es5",
                "removeComments": true,
                "outFile": "../../built/local/tsc.js"
            },
            "files":[
                "app.ts",
                "interfaces.ts",
                "classes.ts",
            ]
        }
\end{verbatim}

Здесь используются все те же параметры, которые применяются при компиляции в
командной строке. Например, параметр $"target"$
указывает, какой стандарт $JavaScript$ будет применяться при компиляции, параметр
$"removeComments"$ удаляет комментарии, параметр $"outFile"$ задает название выходного файла, а
с помощью секции $files$ можно установить набор включаемых в проект файлов.

При необходимости можно включать другие опции компиляции.

Целевой язык программирования имеет систему сборки, абсолютно аналогичную $tsconfig$. Это значит,
что включение опций анализатора можно повторить по аналогии, как в командной строке. Таким образом,
остается только научить преобразовывать написанный в конфигурационном файле текст в активацию
нужных проверок. При добавлении такого раздела вышеуказанный пример изменяется
следующим образом:

\begin{verbatim}
        {
            "compilerOptions": {
                "target": "es5",
                "removeComments": true,
                "outFile": "../../built/local/tsc.js",
                "analizer": ["--enable-all-checks"]
            },
            "files":[
                "app.ts",
                "interfaces.ts",
                "classes.ts",
            ]
        }
\end{verbatim}

То есть просто добавляется строчка в раздел $compilerOptions$, что потребовало добавления
одной функции в реализации компилятора целевого ЯП.

Таким образом, все поставленные в работе задачи выполнены.

\chapter{Заключение}

Обратимся к разделу постановки проблемы и пройдем по каждому выдвинутому пункту.

\begin{enumerate}
    \item Существующие решения изучены и качественно рассмотрены.
    На основании нескольких из них выстроена схема
    работы системы анализа и предупреждений, а именно: $Clang-Tidy$ система и $Standard$ $ECMA$-$327$ послужили основными опорными
    точками при построении решения.
    \item Проведен анализ текущей спецификации целевого языка программирования. Было выделено 7 гипотез для дальнейшего тестирования
    средствами бенчмаркинга. Среди выдвинутых предложений оказались: выявление скрытой упаковки
    и распаковки типов, применение модификатора $final$ для методов и классов при отсутствии наследования,
    запрет инструкций верхнего уровня, перестановка порядка операндов в выражениях равенства,
    использование корутин вместо асинхронных функций, использование вызова функций вместо лямбд,
    замена остаточных параметров на массивы или опциональные параметры.
    \item Для каждой гипотезы представлен вариант потенциального ускорения, заключающийся в тривиальном
    переписывании участка кода, и протестирован с помощью бенчмаркинга.
    \item При бенчмаркинге такие проверки, как имплицитный боксинг и анбоксинг типов, операторы равенства, rest
    параметры показали наиболее значительный прирост в скорости, а проверки на модификатор $final$ в классах
    и методах, использование вызова функций вместо лямбд дали меньшую, но тоже очень заметную разницу
    во времени исполнения. Проверка на замену асинхронных функций корутинами дала прирост порядка $1.5\%$ для
    $int$ типа и показала значительный регресс на тестировании при использовании типа $string$. Эта проверка
    необходима с точки зрения идеологии целевого языка согласно его спецификации, но требует осторожного применения,
    более конкретных замеров с использованием других оптимизирующих инструментов. Проверка на запрет использования
    инструкций верхнего уровня дала уменьшение времени запуска, что также является целевым результатом в рамках
    поставленных целей и задач работы.
    \item Разработана система, анализирующая исходный код согласно включенному набору проверок. Система работает
    с типами узлов, поддерживает режимы предупреждений и ошибки времени компиляции. Количество проверок составляет
    семь наименований, причем возможно групповое включение с учетом совместимости и несовместимости с
    $TypeScript$, а также включение всех проверок сразу одной опцией компиляции.
    \item Разработана система-аналог $Clang-Tidy$, позволяющая точечно в исходном коде контролировать,
    какие участки кода следует исключить из проверяющей системы. Поддержаны те же типы специализированных
    комментариев, что и в $Clang-Tidy$, а именно: отключение для текущей строки, для следующей строки и отключение
    блока кода, заключенного между комментариями, обозначающими начало и конец подавляющей секции.
    \item Поддержана возможность анализа в многофайловой сборке через специальный файл, задающий
    конфигурацию проекта.
\end{enumerate}

В анализирующую систему включено 7 проверок, которые в среднем дают ускорение более, чем на 10\%
при замере бенчмарками.

Таким образом, поставленные задачи выполнены и цели работы были успешно
достигнуты.

\backmatter

\chapter{Литература}
\begin{thebibliography}{99}
    \bibitem{ecma-327}
    ECMA International. Standard ECMA-327: ECMAScript 3rd Edition Compact Profile, June 2001
    \bibitem{eng-article}
    Rajaraman, V.[et al]. Programming languages //
    Resonance 3. 1998. DOI: 10.1007/BF02838097
    \bibitem{benchmark-book}
    Aкиньшин, А. Профессиональный бенчмарк: искусство измерения производительности. Библиотека программиста, 2022.
    \bibitem{jvm-spec}
    Oracle. (n.d.). Java Virtual Machine (JVM) documentation.
    URL: https://docs.oracle.com/javase/specs/jvms/se8/html/index.html
    \bibitem{clang-tidy}
    LLVM Project. Clang-Tidy Documentation. LLVM Project,
    URL: https://clang.llvm.org/extra/clang-tidy/ - 2024.
    \bibitem{spark}
    AdaCore. SPARK Language Reference Manual. AdaCore, 2022.
    \bibitem{ravenscar}
    AdaCore. Ravenscar Profile for High-Integrity Systems // AdaCore, 2005.
    \bibitem{misra}
    MISRA. MISRA C++:2008 Guidelines for the Use of the C++ Language in Critical Systems. MISRA - 2008.
    \bibitem{cert}
    CERT Coordination Center. CERT C Coding Standard. CERT Coordination Center - 2022.
    \bibitem{java-me}
    Oracle. "Java Platform, Micro Edition (Java ME) Documentation." Oracle, URL: https://www.oracle.com/java/technologies/java-me.html - 2024.
    \bibitem{java-se}
    Oracle. "Java Platform, Standard Edition (Java SE) Documentation." Oracle, URL: https://docs.oracle.com/javase/ - 2024.
    \bibitem{tsspec}
    TypeScript. (n.d.). TypeScript: Handbook. TypeScript, n.d. - 2024.
    \bibitem{sun-test}
    Curran, Patrick. Conformance Testing: An Industry Perspective. Sun Microsystems - 2003
    \bibitem{pierce}
    Pierce, Benjamin C. 2002. Types and Programming Languages. Cambridge, MA: MIT Press.
    \bibitem{spec-language}
    Massachusetts Institute of Technology. (n.d.). Introduction to Spec. URL: http://web.mit.edu/www/notes
    \bibitem{dragon-book}
    Aho, Alfred V., Monica S. Lam, Ravi Sethi, and Jeffrey D. Ullman. Compilers: Principles, Techniques, and Tools. Pearson Education - 2006.
    \bibitem{js-spec}
    ECMA International. (2020). ECMAScript 2020 Language Specification.
    URL: https://www.ecma-international.org/ecma-262/11.0/index.html
    \bibitem{cplusplus}
    ISO International Standard ISO/IEC 14882:2020(E) – Programming Language C++,
     URL: https://isocpp.org/std/the-standard
\end{thebibliography}

\end{document}